\newcommand{\JournalEntry}[5]{%
    \begin{tabular}{|l l l|}
        \hline
        \multicolumn{3}{|c|}{\textbf{Entry \##1}}\\
        \multicolumn{2}{|l}{\today} & \multicolumn{1}{r|}{John Smith}\\
        \hline
        Account & Debit & Credit\\
        \hline
        #2 &  & \$#4\\
        #3 & \$#5 & \\
        \hline
    \end{tabular}
}

\newcommand{\TChart}[4]{%
    \begin{tabular}{c|c}
        \multicolumn{2}{c}{\makecell{\textbf{#1}\\(#2)}}\\
        \hline
        Debits & Credit\\
        #3 & #4
    \end{tabular}

}

\chapter{Accounting Fundamentals}

This section provides an overview of the accounting basics that I learned in preperation for working on this project.

This was written by Gregory Bell and he has no idea what he is talking about.

For more information, I would checkout the gnucash book as it is \textit{a lot} more detailed and written by actual experts.

I also will be providing examples on how these princpals apply to Mines Little Theatre.

\section{Debits and Credits}

Credits and debits in accounting are different in accounting than they are in everyday lanuage. Infact, they are opposite of their common meaning.

Accounting tracks the flow of \textbf{value}, not necessarily money \cite{NetsuiteDebitsCredit}. 

The fundemental accounting equation $\textit{Assets} = \textit{Liabilities} + \textit{Equity}$ dictates that all debits and credits must be equal \cite{AccountingCoachDebit}.

Finally, there must be an equal number of credits and debits for each transaction \cite{AccountingCoachDebit}. 

This concept will be explored more in \nameref{DoubleEntryAccounting}.

\subsection{Example}
Generally, accounts are displayed in a T Chart. \\

\begin{tabular}{C{.33\linewidth}|L{.66\linewidth}}
    \TChart{Cash}{Asset}{Increase}{Decrease} & This T chart shows how an asset account (in this case cash) is setup to increase value with a debit and decrease value with a credit \cite{AccountingCoachDebit}.\\
    \TChart{Rent}{Expense}{Increase}{Decrease} & This T chart shows how an expense account (which is an asset) is setup to increase value with a debit and decrease value with a credit. Think about how the liability would increase as this expense increases \cite{AccountingCoachDebit}.\\
    \TChart{Reveune}{Equity}{Decrease}{Increase} & This T chart shows how reveune (which is equity) decreases with a debit, and increases with a credit \cite{NetsuiteDebitsCredit}.
\end{tabular}

\clearpage

\section{The Journal}

The journal is a record of all the transactions that a business makes.

Every time that a transaction is made, a entry is entered into the journal.

There are two primary types of journal entry methods, single entry, where each transaction is entered once as either a credit or a debit \cite{InvestopiaJournal}. 
This is very simlar to how one may record transactions in a check book \cite{InvestopiaJournal}. It may be helpful to think of single entry accounting as the naive approach to accounting.
During the 2023 - 2024 school year, Gregory Bell used a single entry journal for Mines Little Theatre.

The other type of journal entry method is called double entry, where each transaction is entered as \textbf{both} a credit and a debit such that they sum to 0 \cite{InvestopiaJournal}.
This journal entry method is a lot more robust, but is a lot more of a pain to maintain espcially given the already high time committment of MLT \cite{InvestopiaJournal, NetsuiteJournal}.

Entries update the general ledger (techincally a system of ledgers where each account has its own ledger), which in turn updates the accounts balances \cite{NetsuiteJournal}.

Becuase the journal tracks all of the financial activity for an organizaion, it is the underpinning of the rest of the accounting system.

\subsection{Journal Entry}

When using double entry accounting, each journal must have and equal number of debits and credits.

For the purposes of MLT (being that we are a school organizaion) there is only one debit and credit per entry.

This is a naive approach to double entry accounting, but I don't think that there is need for a more complex system.

\begin{tabular}{C{.5\linewidth}|L{.5\linewidth}}
    \JournalEntry{00000}{Accounts Payable}{Rent}{100}{1000} & Notice how this entry is \textbf{not} in balance. There are an unqual number of credits to debits.\\
    \\
    \JournalEntry{00001}{Accounts Payable}{Rent}{1000}{1000} & Notice how this entry \textbf{is} in balance. This entry would be posted to the general ledger at the end of the accounting period.\\
\end{tabular}

\section{Double Entry Accounting} \label{DoubleEntryAccounting}

As mentioned above, double entry accounting is a more robust way to record financial transactions. This is due to each transaction entering \textbf{both} a credit and debit \textbf{and} requiring that they sum to 0.

This ensures that the fundamental accounting equation, $\textit{Assets} = \textit{Liabilities} + \textit{Equity}$, remains balanced.

One of the primary benifits of using double entry accounting is that it makes auditing the books a lot simpler. It also allows us to have more transparency where money is coming from, and where money is going \cite{NetsuiteJournal}.

Going forward, it is in Mines Little Theatre's best interest to use a double entry accounting system, espcially as our allocations get larger and our show budgets become more complex.

Personally, I find it really hard to understand how double entry accounting works and how debiting and crediting work. The following examples will hopefully illuminate how double entry accounting works with in the context of Mines Little Theatre.

\subsection{Example}
Let the chart of accounts look like this:\\

\begin{center}
    \begin{tabular}{cc}
        \TChart{Ticket Sales}{Equity}{Decrease}{Increase} & \TChart{Cash}{Asset}{Increase}{Decrease}
    \end{tabular}
    \vspace{1cm}

    \begin{tabular}{r|ll}
        Account & Debit & Credit\\
        \hline
        Ticket Sales & & \$500.00\\
        Cash & \$500.00 & \\
    \end{tabular}
\end{center}


\begin{gather*}
    \textit{Assets} = \textit{Liabilities} + \textit{Equity}\\
    \textit{Assets} = \mathlarger{\mathlarger{\sum}}\textit{total balances} = \text{\$}500.00\\
    \textit{Liabilities} = \mathlarger{\mathlarger{\sum}}\textit{total balances} = \text{\$}0.00\\
    \textit{Equity} = \mathlarger{\mathlarger{\sum}}\textit{total balances} = \text{\$}500.00\\
    \text{\$}500.00 = 0 + \text{\$}500.00
\end{gather*}

As you can see, the accounting equation is balanced before we we enter a new journal entry.

Thus, a journal entry for this chart of accounts using double entry accounting would look like:\\

\begin{tabular}{C{.5\linewidth}|L{.5\linewidth}}
    \JournalEntry{0001}{Ticket Sales}{Cash}{100}{100} & This example shows how a journal entry would be entered for \$100 earned in ticket sales.\\
    &\\
    \JournalEntry{0002}{Cash}{Ticket Sales}{10}{10} & This example shows how a journal entry would be entered for a \$10 refund.
\end{tabular}

\begin{center}
    \begin{tabular}{r|ll}
        Account & Debit & Credit\\
        \hline
        Ticket Sales & & \$590.00\\
        Cash & \$590.00 & \\
    \end{tabular}
\end{center}

Notice how our total cash increased, but the accounting equation is still equal to zero.


\section{The Ledger}

The ledger is an imutable source of truth for each financial account built from the journal when journal entries are ``posted''.

MLT has labeled many things as a ``ledger'' in the past, but they functioned more like a journal. This isn't wrong persay, but it does mean that the source of truth can change which isn't ideal.

In order to ensure that MLT's finances remain as accurate as possible, an effective ledger must be created

